\documentclass[11pt]{article}

\usepackage[a4paper, total={15cm, 20cm}]{geometry}
\usepackage[ngerman]{babel}
\usepackage{fancyhdr}

\pagestyle{fancy}
\fancyhead{}
\fancyfoot{}
\fancyhead[L]{\slshape \MakeUppercase{Diplomarbeit Antrag}}
\fancyhead[R]{\slshape \MakeUppercase{5AHEL 2020/21}}
\fancyfoot[C]{\thepage}
\setlength{\headheight}{15pt}

\parindent 0ex
\renewcommand{\baselinestretch}{1.5}

\begin{document}

\begin{titlepage}
  \begin{center}
    \vspace*{1cm}
    \large{\textbf{HTL Rankweil}}\\
    \large{\textbf{5AHEL 2020/21}}
    \vfill
    \line(1,0){400}\\[1mm]
    \huge{\textbf{Diplomarbeit Antrag}}\\[3mm]
    \large{\textbf{Parkplatz-Freimeldesystem}}\\[1mm]
    \line(1,0){400}\\
    \vfill
    Philipp Kraft, Dennis Köb und Samuel Brugger\\[3mm]
    \today, Rankweil
  \end{center}
\end{titlepage}

\tableofcontents
\thispagestyle{empty}
\clearpage

\setcounter{page}{1}
\section{Diplomarbeit Antrag 2020/21}

\subsection{Name der Arbeit}
Parkplatz-Freimeldesystem

\subsection{Abgabetermin}
Die Abgabe von diesem Antrag ist am 19.05.2020 vorgesehen.

\subsection{Schule und Abteilung}
Höhere Technische Bundeslehr- und Versuchsanstalt Rankweil\\
Negrellistraße 50, A-6830 Rankweil\\
Schulleiterin: Mag. Zeiner-Mohr Judith\\

Elektronik und Technische Informatik\\
Abeitlungsvorstand: Dipl.-Ing. Moosbrugger Leopold

\subsection{Ausgangslage}
Die Parkplatzsuche in Städten verursacht beträchtliche Zeitverluste und eine untragbare Umweltbelastung. Das zu entwickelnde System verfügt über einen Schwenk-/Neigekopf mit aufmontiertem Laserabstandssensor. Es scannt Parkplatz für Parkplatz und prüft, ob er mit einem Fahrzeug besetzt ist oder nicht. Die Daten werden per WLAN an eine Zentralstation gesendet. Die Zentralstation sendet die Anzahl der freien Parkplätze an ein oder mehrere Anzeigeeinheiten. Eine komfortable Eingabemöglichkeit der Parkplatzpositionen ist vorzusehen.

\subsection{Individuelle Themenstellungen}
Für eine ausführliche Individuelle Themenstellung ist es zu früh, daher ist diese nur unspezifisch angeführt.

\begin{table}[htb]
  \begin{tabular}{|l|l|l|}
    \hline
    \textbf{Vor- und Nachname} & \textbf{Individuelle Themenstellung} & \textbf{Klasse} \\ \hline
    Philipp Kraft              & Projektmanagment/Software            & 5AHEL           \\ \hline
    Dennis Köb                 & Hardware                             & 5AHEL           \\ \hline
    Samuel Brugger             & Software/Hardware                    & 5AHEL           \\ \hline
  \end{tabular}
\end{table}

\subsection{Beteiligte Betreuer/innen}
Dipl.-Ing. Stüttler Christoph 

\subsection{Beteiligte Kooperationspartner/innen}
Derzeit sind keine Kooperationspartner/innen vorhanden.

\subsection{Rechtliche Regelung}
Die Rechtliche Regelung erfolgt durch die HTL Rankweil.

\subsection{Zielsetzung}
Es ist ein System zu entwerfen, mit dem es möglich ist fesgelegte Parkplätze automatisch zu scannen und somit zu erkennen ob dieser Parkplatz besetzt ist.

\subsection{Kurzfassung/Abstract}
Lorem ipsum dolor sit amet, consetetur sadipscing elitr, sed diam nonumy eirmod tempor invidunt ut labore et dolore magna aliquyam erat, sed diam voluptua. At vero eos et accusam et justo duo dolores et ea rebum. Stet clita kasd gubergren, no sea takimata sanctus est Lorem ipsum dolor sit amet. Lorem ipsum dolor sit amet, consetetur sadipscing elitr, sed diam nonumy eirmod tempor invidunt ut labore et dolore magna aliquyam erat, sed diam voluptua. At vero eos et accusam et justo duo dolores et ea rebum. Stet clita kasd gubergren, no sea takimata sanctus est Lorem ipsum dolor sit amet.

\subsection{Typ der Arbeit}
§ 7 Abs. 1 PrüfOrd. BHS, BA definiert: 

„Die Diplomarbeit an höheren Schulen (§ 2 Abs. 4 Z 1 lit. a)
besteht nach Maßgabe des 4. Abschnittes aus einer auf vorwissenschaftlichem Niveau zu erstellenden
schriftlichen Arbeit (bei entsprechender Aufgabenstellung auch unter Einbeziehung praktischer und/oder
grafischer Arbeitsformen) mit Diplomcharakter über ein Thema gemäß § 3 sowie deren Präsentation und
Diskussion.

\end{document}}