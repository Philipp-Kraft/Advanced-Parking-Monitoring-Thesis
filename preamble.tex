%% ==================================================================
%%%% Packages
%% ==================================================================
\usepackage[a4paper]{geometry}
\usepackage[utf8]{inputenc}
\usepackage[ngerman]{babel}
\usepackage[T1]{fontenc}
\usepackage{newpxtext,newpxmath}
\usepackage[hyphens,spaces,obeyspaces]{url}
\usepackage[hidelinks]{hyperref}
\usepackage{amsmath,amssymb,amstext}
\usepackage{mathtools}
\usepackage[backend=biber,citestyle=verbose, bibstyle=verbose]{biblatex}
\usepackage[babel,german=guillemets]{csquotes}
\usepackage{fancyhdr}
\usepackage{microtype}
\usepackage{graphicx}
\usepackage{float}
\usepackage{minted}
\usepackage{acronym}
\usepackage{booktabs}
\usepackage{multirow}
\usepackage{pgf}
%\usepackage{pgfplots}
\usepackage{siunitx}
\usepackage{xcolor,mdframed}
\usepackage{lipsum}
\usepackage{listings,chngcntr} 			% kapitelweise Nummerierung für Codeabschnitte
\usepackage{listings,xcolor} 				% ermöglicht das Einbinden von Codesegmenten
\usepackage{listingsutf8} 					% ermöglicht das Einbinden von Codesegmenten
\usepackage{subfigure}
\usepackage{caption}


%% ==================================================================
%%%% General Document Settings
%% ==================================================================
\linespread{1.5}
\parindent 0ex

\bibliography{bibliography.bib}

\graphicspath{ {./images/} }

\setcounter{tocdepth}{4}
\setcounter{secnumdepth}{4}

\sisetup{locale = DE}

\raggedbottom % Allow ragged page bottoms

\renewcommand{\familydefault}{\sfdefault} % Arial ähnliche Schriftart

\def \sectionauthors {Philipp Kraft, Dennis Köb und Samuel Bleiner}


%% ==================================================================
%%%% Fancy Header Settings
%% ==================================================================
\pagestyle{fancy}
\fancyhead{}
\fancyfoot{}
\fancyhead[L]{\hspace{0.2cm} \includegraphics[height=1.04cm]{htl_rankweil_logo.png}}
\fancyhead[C]{\textbf{HTL Rankweil}
  \\[0.05in]
  \footnotesize{Höhere Lehranstalt für Elektronik und Technische Informatik}}
\fancyhead[R]{\includegraphics[height=1.04cm]{htl_logo.png} \hspace{0.2cm}}

\fancyfoot[LE,RO]{\thepage}
\fancyfoot[LO,RE]{2020/21 Advanced Parking Monitoring}

\setlength{\headheight}{34pt}
\renewcommand{\footrulewidth}{0.4pt}


%% ==================================================================
%%%% Listings Settings
%% ==================================================================
\newenvironment{longlisting}{\captionsetup{type=listing}}{}
\renewcommand{\listingscaption}{Code}
\renewcommand{\listoflistingscaption}{Codeverzeichnis}

\definecolor{backcolour}{gray}{0.95}

\setminted
{
  mathescape,
  bgcolor = backcolour,
  linenos = true,
  numbersep = 5pt,
  gobble = 2,
  framesep = 2mm,
  numberblanklines = true,
  breaklines = true,
  numberblanklines = true
}


%% ==================================================================
%%%% Important Info Block
%% ==================================================================
\newenvironment{important}[1][]{%
  \begin{mdframed}[%
      backgroundcolor={red!15}, hidealllines=true,
      skipabove=0.7\baselineskip, skipbelow=0.7\baselineskip,
      splitbottomskip=2pt, splittopskip=4pt, #1]%
    \makebox[0pt]{% ignore the withd of !
      \smash{% ignor the height of !
        \fontsize{32pt}{32pt}\selectfont% make the ! bigger
        \hspace*{-19pt}% move ! to the left
        \raisebox{-2pt}{% move ! up a little
          {\color{red!70!black}\sffamily\bfseries !}% type the bold red !
        }%
      }%
    }%
    }{\end{mdframed}}


%% ==================================================================
%%%% Frontmatter, Mainmatter and Backmatter
%% ==================================================================
\def\frontmatter{%
  \pagenumbering{roman}
  \setcounter{page}{1}
  \renewcommand{\thesection}{\Roman{section}}
}%

\def\mainmatter{%
  \pagenumbering{arabic}
  \setcounter{page}{1}
  \setcounter{section}{0}
  \renewcommand{\thesection}{\arabic{section}}
}%

\def\backmatter{%
  \setcounter{section}{0}
  \renewcommand{\thesection}{\Alph{section}}
}%

