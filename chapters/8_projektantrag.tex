\subsection{Name der Arbeit}
APM - Advanced Parking Monitoring

\subsection{Abgabetermin}
Die Abgabe von diesem Antrag ist am ??.10.2020 vorgesehen.

\subsection{Schule und Abteilung}
Höhere Technische Bundeslehr- und Versuchsanstalt Rankweil\\
Negrellistraße 50, A-6830 Rankweil\\
Schulleiterin: Mag. Zeiner-Mohr Judith\\

Elektronik und Technische Informatik\\
Abeitlungsvorstand: Dipl.-Ing. Moosbrugger Leopold

\subsection{Ausgangslage}
Die Parkplatzsuche in Städten verursacht beträchtliche Zeitverluste und eine untragbare Umweltbelastung. Das zu entwickelnde System verfügt über einen Schwenk-/Neigekopf mit aufmontiertem Laserabstandssensor. Es scannt Parkplatz für Parkplatz und prüft, ob er mit einem Fahrzeug besetzt ist oder nicht. Die Daten werden per WLAN an eine Zentralstation gesendet. Die Zentralstation sendet die Anzahl der freien Parkplätze an ein oder mehrere Anzeigeeinheiten. Eine komfortable Eingabemöglichkeit der Parkplatzpositionen ist vorzusehen.

\subsection{Individuelle Themenstellungen}
Für eine ausführliche Individuelle Themenstellung ist es zu früh, daher ist diese nur unspezifisch angeführt.

\begin{table}[htb]
  \begin{tabular}{|l|l|l|}
    \hline
    \textbf{Vor- und Nachname} & \textbf{Individuelle Themenstellung} & \textbf{Klasse} \\ \hline
    Philipp Kraft              & Projektmanagment/Software            & 5AHEL           \\ \hline
    Dennis Köb                 & Hardware                             & 5AHEL           \\ \hline
    Samuel Brugger             & Software/Hardware                    & 5AHEL           \\ \hline
  \end{tabular}
\end{table}

\subsection{Beteiligte Betreuer/innen}
Dipl.-Ing. Stüttler Christoph

\subsection{Beteiligte Kooperationspartner/innen}
Derzeit sind keine Kooperationspartner/innen vorhanden.

\subsection{Rechtliche Regelung}
Die Rechtliche Regelung erfolgt durch die HTL Rankweil.

\subsection{Zielsetzung}
Es ist ein System zu entwerfen, mit dem es möglich ist fesgelegte Parkplätze automatisch zu scannen und somit zu erkennen ob dieser Parkplatz besetzt ist.

\subsection{Kurzfassung/Abstract}
Es ist ein System zu entwerfen, welches es ermöglicht öffentliche Parkplätze zu überwachen und verwalten. Das System ist dabei modular und kostengünstig.

\subsection{Typ der Arbeit}
§ 7 Abs. 1 PrüfOrd. BHS, BA definiert:

„Die Diplomarbeit an höheren Schulen (§ 2 Abs. 4 Z 1 lit. a)
besteht nach Maßgabe des 4. Abschnittes aus einer auf vorwissenschaftlichem Niveau zu erstellenden
schriftlichen Arbeit (bei entsprechender Aufgabenstellung auch unter Einbeziehung praktischer und/oder
grafischer Arbeitsformen) mit Diplomcharakter über ein Thema gemäß § 3 sowie deren Präsentation und
Diskussion.