\def \sectionauthors {Philipp Kraft}

\subsection{Anforderungen}
Das Webinterface hat auf der einen Seite die Aufgabe die Kommunikation mit der
Kennzeichenerkennung und der Induktionmessung sicherzustellen und auf der
anderen Seite die Verwaltung und Darstellung der gewonnen Daten.


\subsection{Lokale Entwicklungsumgebung mit Laragon}
Für die Programmierung des Webinterfaces müssen zuerst einige Vorkehrungen
getroffen werden, dazu zählt zu einem die Installation von benötigter Software
und deren konfiguration.


\subsubsection{Benötigte Software}

\begin{itemize}
  \item \textbf{Laragon} (\url{https://laragon.org}) \\Beinhaltet mehrere
        Softwarepakete die für die Entwicklung notwendig sind.
        \begin{itemize}
          \item Apache HTTP Server
          \item MySQL
          \item PHP
        \end{itemize}
  \item \textbf{phpMyAdmin} (\url{https://www.phpmyadmin.net}) \\ Webinterface
        für MySQL
  \item \textbf{Composer} (\url{https://getcomposer.org}) \\ Paketmanager für
        PHP
  \item \textbf{Git} (\url{https://git-scm.com}) \\ Versionskontrolle
  \item \textbf{Visual Studio Code} (\url{https://code.visualstudio.com}) \\
        Quelltext-Editor
\end{itemize}


\subsubsection{Konfiguration von PHP}
Um PHP Befehle von der Kommandozeile auszuführen muss die Installation zuerst in
den Windows Path Variables hinzugefügt werden.

Dies erfolgt durch die \verb|Advanced System Settings| $\blacktriangleright$
\verb|Environment Variables| $\blacktriangleright$ \verb|System Variables|.
Dort kann nun die Path Variable editiert werden und der Pfad hinzugefügt werden
in welchem die \verb|php.exe| liegt.

\begin{figure}[H]
  \centering
  \includegraphics[width=0.6\linewidth]{webinterface/advanced_system_settings.png}
  \caption{Advanced System Settings}
\end{figure}

\begin{figure}[H]
  \centering
  \includegraphics[width=0.6\linewidth]{webinterface/system_variables.png}
  \caption{System Variables}
\end{figure}

\begin{figure}[H]
  \centering
  \includegraphics[width=0.6\linewidth]{webinterface/environment_variable.png}
  \caption{Environment Variables}
\end{figure}

Die korrekte konfiguration kann durch die Kommandozeile geprüft werden, dort
muss das Befehl \verb|php -v| ausgeführt werden. Dabei ist zu beachten, dass
nach dem hinzufügen der Path Variable die gewählte Kommandozeile neu gestartet
werden muss.

\begin{figure}[H]
  \centering
  \includegraphics[width=1\linewidth]{webinterface/php_version.png}
  \caption{PHP Version}
\end{figure}

Somit ist PHP korrekt konfiguriert.


\subsubsection{Installation von phpMyAdmin}
phpMyAdmin ist ein Tool, welches den Umgang mit MySQL Datenbanken mit einem
Webinterface erleichtert. Die aktuellste Version lässt sich von
\url{https://www.phpmyadmin.net/downloads} downloaden. Dieses Archiv muss
entpackt werden und ausgehend vom Laragon Root Verzeichnis in das Verzeichnis
\verb|/etc/apps| kopiert werden. Um die Installation zu überprüfen muss der
Apache HTTP Server und der MySQL Server gestartet werden, nun sollte bei einer
korrekten Installation das Webinterface von phpMyAdmin unter
\url{http://localhost/phpmyadmin} erreichbar sein.

\begin{figure}[H]
  \centering
  \includegraphics[width=0.5\linewidth]{webinterface/phpmyadmin.png}
  \caption{phpMyAdmin Webinterface}
\end{figure}

Es ist nicht notwendig ein Passwort einzugeben, da Standardmäßig kein Passwort
gesetzt wird.


\subsection{Lokale Entwicklungsumgebung mit WSL und Docker}


\subsubsection{Benötigte Software}

\begin{itemize}
  \item \textbf{Docker} (\url{https://www.docker.com}) \\ Ermöglicht Isolation
        von Anwendungen mit Containervirtualisierung
  \item \textbf{WSL} (\url{https://docs.microsoft.com/en-us/windows/wsl}) \\
        Kompatibilitätsschicht für Linux Anwendungen unter Windows 10
  \item \textbf{Visual Studio Code} (\url{https://code.visualstudio.com}) \\
        Quelltext-Editor
\end{itemize}


\subsubsection{Installation von WSL}
Zuerst müssen einige Einstellungen in Windows getroffen werden um später eine
Linux Distribution herunterzuladen können. Diese Befehle können über die
Kommandozeile mit Administrativen Rechten ausgeführt werden.

\paragraph{1. Schritt: WSL Aktivieren}\mbox{}\\
\begin{lstlisting}[caption={WSL Feature Feature aktivierens}]
  dism.exe /online /enable-feature /featurename:Microsoft-Windows-Subsystem-Linux /all /norestart
\end{lstlisting}

\paragraph{2. Schritt: Virtual Machine Aktivieren}\mbox{}\\
\begin{lstlisting}[caption={Virtual Machine Feature aktivieren}]
  dism.exe /online /enable-feature /featurename:VirtualMachinePlatform /all /norestart
\end{lstlisting}

Nach diesem Schritt ist ein Neustart des Computers notwendig.

\paragraph{3. Schritt: Linux Kernel Update}\mbox{}\\
Nun muss ein Linux Kernel Update installiert werden, die aktuelle Version ist
unter \url{https://aka.ms/wsl2kernel} zu finden.

\paragraph{4. Schritt: WSL 2}\mbox{}\\
Nach dem Neustart des Computers sollte es nun möglich sein WSL 2 als Version
auszuwählen.
\begin{lstlisting}[caption={WSL 2 auswählen}]
  wsl --set-default-version 2
\end{lstlisting}

\paragraph{5. Schritt: Linux Distribution herunterladen}\mbox{}\\
Zuletzt kann eine Linux Distribution aus dem Windows Store heruntergeladen
werden, in diesem Fall Debian (\url{https://www.microsoft.com/de-de/p/debian}).


\subsubsection{Installation von Docker}
Die aktuellste Version von Docker Desktop für Windows lässt sich am einfachsten
über die offizielle Website von Docker herunterladen (\url{https://docker.com}).
Nach der Installation muss noch die WSL Integration aktiviert werden. Dazu muss
in den Einstellungen unter \verb|Resources| $\blacktriangleright$ \verb|WSL Integration| und
dort muss der Haken bei \verb|Enable integration with my default WSL distro| gesetzt
werden und die installierte Linux Distribution muss unten aktiviert werden.

\begin{figure}[H]
  \centering
  \includegraphics[width=1\linewidth]{webinterface/docker.png}
  \caption{Docker WSL Integration}
\end{figure}

Somit ist die Lokale Entwicklungsumgebung mit WSL und Docker abgeschlossen, die
benötigte Software wird später automatisch durch Laravel Sail in einem Docker
Container installiert.

\begin{figure}[H]
  \centering
  \includegraphics[width=1\linewidth]{webinterface/docker_container.png}
  \caption{Docker Container Steuerung}
\end{figure}

Es ist somit möglich die Services welche im Container in der Linux Distribution
laufen über die Docker Desktop Anwendung zu steuern.


\subsection{Production Server}
Der Production Server bzw. der Live Server ist der Server wo sich die Webanwendung
befindet und die Endbenutzer zugreifen, dieser Server wird auch einfach mit
Production abgekürzt. Der Production Server ist in diesem Fall ein Virtual
Private Server mit dem Betriebsystem Debian 10, welcher bei einem
Internet-Hosting Unternehmen mit Sitz in Deutschland gehostet wird.


\subsubsection{Benötigte Software}

Für den Live Server wird der sogenannte \glqq LAMP\grqq{} Stack verwendet. LAMP steht dabei
für die Software \textbf{L}inux, \textbf{A}pache, \textbf{M}ySQL und \textbf{P}HP.

\begin{itemize}
  \item \textbf{Apache Web Server} (\url{https://httpd.apache.org}) \\ HTTP Server
  \item \textbf{MariaDB} (\url{https://mariadb.org}) \\ Fork von MySQL
  \item \textbf{PHP} (\url{https://mariadb.org}) \\ Fork von MySQL
  \item \textbf{phpMyAdmin} (\url{https://www.phpmyadmin.net}) \\ Webinterface
        für MySQL
  \item \textbf{Composer} (\url{https://getcomposer.org}) \\ Paketmanager für PHP
  \item \textbf{Git} (\url{https://git-scm.com}) \\ Versionskontrolle
        Quelltext-Editor
\end{itemize}


\subsubsection{Installation des LAMP Stacks}
Bevor die Software Pakete installiert werden sollte die Linux Software/Update
Repository geupdatet werden.

\begin{lstlisting}[caption={Respositorys updaten}]
  apt-get update && apt-get upgrade
\end{lstlisting}

\paragraph{Apache}\mbox{}\\

Nun kann der Apache Web Server installiert werden.

\begin{lstlisting}[caption={Apache installieren}]
  apt install apache2
\end{lstlisting}

Die Installation kann nun leicht überprüft werden indem man im Browser die IP
bzw. die dazugehörige Domain öffnet, in diesem Fall:
(\url{http://dev.philipp-kraft.com}).

\begin{figure}[H]
  \centering
  \includegraphics[width=1\linewidth]{webinterface/apache2_installation.png}
  \caption{Debian Default Page}
\end{figure}

Erscheint die Debian Default Page ist Apache korrekt installiert.

\paragraph{PHP}\mbox{}\\
Neben PHP werden auch einige PHP Extensions benötigt.
\begin{lstlisting}[caption={PHP installieren}]
  apt install wget php php-cgi php-mysqli php-pear php-mbstring
  php-gettext libapache2-mod-php php-common php-phpseclib php-mysql
\end{lstlisting}

Die Installation kann einfach mit dem Befehl \verb|php -v| überprüft werden.

\paragraph{MariaDB}\mbox{}\\

\begin{lstlisting}[caption={MariadB installieren}]
  apt install mariadb-server
\end{lstlisting}

Nun muss MariaDB noch konfiguriert werden.

\begin{lstlisting}[caption={MariaDB Secure Installation}]
  mysql_secure_installation
\end{lstlisting}

Dabei wird dem root MySQL User ein Passwort gesetzt, Anonyme Benutzer gelöscht
und es werden Test Datenbanken gelöscht.

Nun wird ein neuer Benutzer mit root Berechtigungen erstellt.

\begin{lstlisting}[caption={MariaDB konfiguration}]
  mysql
  GRANT ALL ON *.* TO 'admin'@'localhost' IDENTIFIED 
  BY 'password' WITH GRANT OPTION;
  flush privileges;
  exit
\end{lstlisting}

\paragraph{phpMyAdmin}\mbox{}\\

Die aktuelle Version von phpMyAdmin kann von
(\url{https://www.phpmyadmin.net/downloads}) bezogen werden und mit dem
\verb|wget| Befehl heruntergeladen werden.

\begin{lstlisting}[caption={phpMyAdmin Download}]
  wget https://files.phpmyadmin.net/phpMyAdmin/5.0.4
  /phpMyAdmin-5.0.4-all-languages.tar.gz
\end{lstlisting}

Anschließend muss das Archiv entpackt werden.

\begin{lstlisting}[caption={phpMyAdmin Entpacken}]
  tar xvf phpMyAdmin-5.0.4-all-languages.tar.gz
\end{lstlisting}

Als nächstes muss das entpackte Archiv in einen anderen Pfad verschoben werden
und zusätzlich müssen einige Rechte und Verzeichnisse angepasst werden.

\begin{lstlisting}[caption={phpMyAdmin Rechte und Verzeichnisse}]
  mv phpMyAdmin-5.0.4-all-languages /usr/share/phpmyadmin
  mkdir -p /var/lib/phpmyadmin/tmp
  chown -R www-data:www-data /var/lib/phpmyadmin
  mkdir /etc/phpmyadmin/
\end{lstlisting}

Nun muss eine Konfigurations Datei erstellt werden und dort muss ein Blowfish
Secret\footnote{32 Zeichen String für Cookie-Authentifizierung} angegeben werden und den Pfad für ein Temporäres Verzeichnis.

\begin{lstlisting}[caption={phpMyAdmin Konfigurationsdatei erstellen}]
  cp /usr/share/phpmyadmin/config.sample.inc.php /usr/share/phpmyadmin/config.inc.php
\end{lstlisting}

und am Ende dieser Datei müssen folgende zwei Zeilen eingefügt werden.

\begin{lstlisting}[caption={phpMyAdmin Blowfish Secret und TempDir}]
  $cfg['blowfish_secret'] = 'H2OxcGXxflSd8JwrwVlh6KW6s2rER63i'; 
  $cfg['TempDir'] = '/var/lib/phpmyadmin/tmp';
\end{lstlisting}

Zuletzt muss der Apache Web Server konfiguriert werden.

Im Verzeichnis \verb|/etc/apache2/sites-available| muss eine neue
Konfiguration angelegt werden \verb|phpmyadmin.conf|.

\begin{lstlisting}[caption={phpmyadmin.conf}]
  Listen 9000

  <VirtualHost *:9000>
          ServerName localhost
  
          <Directory /usr/share/phpmyadmin>
                  AllowOverride None
                  Require all granted
          </Directory>
  
          DocumentRoot /usr/share/phpmyadmin
  
          ErrorLog ${APACHE_LOG_DIR}/phpmyadmin.error.log
          CustomLog ${APACHE_LOG_DIR}/phpmyadmin.access.log combined
  </VirtualHost>
\end{lstlisting}

Nun kann diese Virtual Host Konfigurations Datei aktiviert werden und danach
muss der Apache Web Server neu gestartet werden.

\begin{lstlisting}[caption={Virtual Host aktivieren}]
  a2ensite phpmyadmin
  systemctl restart apache2
\end{lstlisting}

Diese Konfiguration
ermöglicht es, dass das Webinterface von phpMyAdmin über den Port 9000 (\url{http://dev.philipp-kraft.com:9000})
erreichbar ist und nicht wie Standardmäßig vorgesehen über das Verzeichnis
/phpmyadmin (\url{http://dev.philipp-kraft.com/phpmyadmin}), dies bietet einen Sicherheitsvorteil.

\paragraph{Webinterface Virtual Host}\mbox{}\\

Nun muss noch eine Virtual Host Konfiguration für das Webinterface selbst erstellt werden.

\begin{lstlisting}[caption={apm.conf}]
  <VirtualHost *:80>
	ServerAdmin webmaster@localhost
	DocumentRoot /var/www/apm/public
        
	<Directory />
		Options FollowSymLinks
		AllowOverride All
	</Directory>

	<Directory /var/www/apm>
		Options Indexes FollowSymLinks MultiViews
		AllowOverride All
		Order allow,deny
		allow from all
	</Directory>

	ErrorLog ${APACHE_LOG_DIR}/error.log
	CustomLog ${APACHE_LOG_DIR}/access.log combined
</VirtualHost>
\end{lstlisting}

Diesmal wird auf den Standard HTTP Port 80 gehört und dieser führt in das Verzeichnis \verb|/var/www/apm/public|. Zusätzlich werden noch Directives gesetzt, damit das Standard \verb|.htaccess| File von Laravel richtig funktionieren kann.

\paragraph{Installation von Composer}\mbox{}\\

Die Installation von Composer gestaltet sich relativ einfach.

\begin{lstlisting}[caption={Download Composer Installer}, language=bash]
  wget -O composer-setup.php https://getcomposer.org/installer
\end{lstlisting}

Nun muss das Setup ausgeführt werden und damit das \verb|composer| Befehl Global verfügbar ist wird Composer in den Pfad \verb|/usr/local/bin| verschoben.

\begin{lstlisting}[caption={Composer Setup}, language=bash]
  php composer-setup.php --install-dir=/usr/local/bin --filename=composer
\end{lstlisting}


\subsubsection{Deployment mit Github Actions}
Unter Deployment versteht man die automatische Installation von Software, in diesem Fall auf einem Linux Server. Erreicht wird das durch zwei Bash Scripts und mit Github Actions (\url{https://github.com/features/actions}).

\paragraph{Git Setup}\mbox{}\\

Sollte auf dem Server noch kein Git installiert sein, lässt sich das wie folgt
installieren.

\begin{lstlisting}[language=bash, caption={Git Installation}]
  apt install git
\end{lstlisting}

Im Verzeichnis \verb|/var/www/apm| soll sich später das Webinterface befinden,
deshalb muss in diesem Pfad Git konfiguriert werden. Dazu wird die Remote URL
konfiguriert.

\begin{lstlisting}[language=bash, caption={Git Remote Origin}]
  git config remote.origin.url 'https://{TOKEN}@github.com/
  Philipp-Kraft/Advanced_Parking_Monitoring_Webinterface.git'
\end{lstlisting}

Da beim Github Account eine Zwei-Faktor-Authentisierung verwendet wird muss die
Authentifizierung mit einem Personal access token erfolgen. Dieser kann unter
\url{https://github.com/settings/tokens} erstellt werden, der erstellte Token
kann dann einfach in der URL eingefügt werden.

\paragraph{Deploy Script}\mbox{}\\

Das Deploy-Script wird auf der Lokalen Entwicklermaschine ausgeführt. Das Script wechselt in den Production Branch und merged mit dem Main Branch, dieser Push in den Production Branch löst dann die Github Action aus.

\begin{lstlisting}[language=bash, caption={phpmyadmin.conf}]
  #!/bin/sh
  set -e
  
  #vendor/bin/phpunit
  
  (git push) || true
  
  git checkout production
  git merge main
  
  git push origin production
  
  git checkout main
\end{lstlisting}

\paragraph{Server Deploy Script}\mbox{}\\

Das Server Deploy Script \verb|server_deploy.sh| versetzt die Laravel Applikation in den Wartungsmodus und lädt sich vom deploy Branch den Code auf den Server herunter, danach werden einige Befehle ausgeführt.

\begin{lstlisting}[language=bash, caption={serverdeploy.sh}]
  #!/bin/sh
  set -e
  
  echo "Deploying application ..."
  
  # Enter maintenance mode
  php artisan down
      
      # Update codebase
      git fetch origin deploy
      git reset --hard origin/deploy
  
      # Install dependencies based on lock file
      composer install --no-interaction --prefer-dist --optimize-autoloader
  
      # Migrate database
      php artisan migrate:refresh --seed
  
      # Clear cache
      php artisan optimize
  
  # Exit maintenance mode
  php artisan up
  
  echo "Application deployed!"
\end{lstlisting}


\paragraph{Github Action}\mbox{}\\
Github Actions ist ein Projekt von Github, welches es ermöglicht Automatisierungen in den Bereichen Entwicklung, Testing und Deployment durchzuführen.

Als erstes muss ein Workflow erstellt werden, dieser wird im Root-Verzeichnis des Projekts erstellt \verb|APM\.github\workflows\main.yml|.

\begin{lstlisting}[caption={main.yml}]
  name: Deploy Laravel app

  on:
    push:
      branches: [ production ]
  
  jobs:
    deploy:
      runs-on: ubuntu-latest
      steps:
      - uses: actions/checkout@v2
        with:
          token: ${{ secrets.PUSH_TOKEN }}
      - name: Set up Node
        uses: actions/setup-node@v1
        with:
          node-version: '12.x'
      - run: npm install
      - run: npm run production
      - name: Commit built assets
        run: |
          git config --local user.email "action@github.com"
          git config --local user.name "GitHub Action"
          git checkout -B deploy
          git add -f public/
          git commit -m "Build front-end assets"
          git push -f origin deploy
      - name: Deploy to production
        uses: appleboy/ssh-action@master
        with:
          username: root
          host: dev.philipp-kraft.com
          password: ${{ secrets.SSH_PASSWORD }}
          script: 'cd /var/apm && ./server_deploy.sh && chown -R www-data.www-data /var/apm && chmod -R 755 /var/apm && chmod -R 777 /var/apm/storage' 
\end{lstlisting}

Dieser Workflow setzt einen Ubuntu Server in der Cloud auf und baut dort die
Assets zusammen, damit die Downtime auf dem Production Server möglichst gering
ist. Danach wird eine SSH Verbindung zum Production Server aufgebaut und dort
wird das Bash-Script \verb|server_deploy.sh| ausgeführt. Gleichzeitig werden
einige Berechtigungen angepasst.

In der Repository muss nun noch das SSH-Passwort und der Personal access token
hinterlegt werden. Dies geschieht über \verb|Settings| $\blacktriangleright$ \verb|Secrets|. Dort
können nun über \verb|New respository secrets| die Secrets hinterlegt werden.

\begin{figure}[H]
  \centering
  \includegraphics[width=1\linewidth]{webinterface/action_secrets.png}
  \caption{Action Secrets}
\end{figure}

\begin{figure}[H]
  \centering
  \includegraphics[width=1\linewidth]{webinterface/github_deploy.png}
  \caption{Github Action Übersicht}
\end{figure}


\subsection{Grundlegender Aufbau}

Da das Ziel ist, dass das Frontend\footnote{Präsentationsebene in Form der
grafischen Benutzeroberfläche} des Webinterfaces möglichst Modular aufgebaut ist
und so wenig Code wie möglich wiederholt wird. Ermöglicht wird dies durch das aufbauen
der Seite mithilfe von Components.


\subsubsection{Components}

Components sind praktisch kleine Bausteine aus denen die komplette Seite
aufgebaut ist. Components sind kein natives Feature von HTML/CSS oder PHP, diese
Funktion wird von der Template Engine Blade bereitgestellt, deshalb werden diese
auch oft Blade Components genannt. 

\paragraph{Anonymous Components}\mbox{}\\

Es gibt viele verschiedene Möglichkeiten
Components zu erstellen und verschiedene Konventionen am, einfachsten sind aber
die \verb|Anonymous Components|, diese haben den Vorteil, dass diese in einer Datei
verwaltet werden können und somit sehr einfach zu handhaben sind.

\paragraph{Components erstellen}\mbox{}\\

Das Erstellen von einem Component wird nun anhand eines Buttons gezeigt. Da
dieser als Anonymous Component angelegt wird muss dieser mit keiner Klasse
assoziiert werden, es wird einfach im Pfad
\verb|resources\views\components| ein Ordner mit dem Namen \verb|buttons|
angelegt und darin ein Blade File mit dem Namen \verb|primary.blade.php|. Dort
kann nun der gewünschte HTML Code platziert werden.

\begin{lstlisting}[caption={primary.blade.php}, label={lst:primary.blade.php}]
  <button type="submit" class="inline-flex items-center px-4 py-2 bg-apm-blue...">
    {{ $slot }}
  </button>
\end{lstlisting}


Im Code~\ref{lst:primary.blade.php} ist eine Variable mit dem Namen \verb|slot|
verwendet worden. Diese Variable wird später beim verwenden automatisch mit dem
Inhalt zwischen dem HTML Element ersetzt.


\paragraph{Components verwenden}\mbox{}\\

Es stellt sich nun die Frage wie man dieses erstelle Component nun verwendet.
Die Blade Components verwenden den gleichen Syntax wie ein normales HTML
Element, mit dem Unterschied dass ein \verb|x-| vor dem Namen des Components
angeführt werden muss. Da sich der vorhin erstellte Component in einem Ordner befindet muss das auch
angegeben werden, dabei wird kein Slash wie üblich um einen Pfad anzugeben
verwendet sondern ein Punkt, es muss auch keine Extensions angegeben werden.

\begin{lstlisting}[caption={Verwendung eines Button Components}]
  <x-buttons.primary>Press me!</x-buttons.primary>
\end{lstlisting}


\paragraph{Attribute übergeben}\mbox{}\\

Auch wenn viele Components ohne Problem überall ohne Veränderung verwendet
werden können, jedoch ist es gewünscht bei manchen Components beispielweise eine
zusätzliche Klasse anzugeben um die Größe des Elements zu verändern. Erreicht
wird das mit der \verb|attributes| Variable im Blade File des Components, da
aber oft schon Attribute definiert sind ist es möglich mit der \verb|merge|
Methode die Attribute zusammenzuführen.

\begin{lstlisting}[caption={Modularer Button Component}]
  <button {{ $attributes->merge(['type' => 'submit], 'class' => 'inline-flex items-center px-4 py-2 bg-apm-blue...') }}>
    {{ $slot }}
  </button>
\end{lstlisting}

Somit ist dieser Button Component nun vollständig Modular.


\subsubsection{Layouts}

Da auf den meisten Seiten des Webinterfaces fast das gleiche Layout beibehaltet ist es sinnvoll diesen Inhalt in ein Component umzuwandeln. Auch Layouts
sind Components.

Es gibt im Webinterface folgende Layouts:

\begin{itemize}
  \item \textbf{admin.blade.php}\\
  Layout für die Administrativen Seiten mit einer Sidebar und Page Header
  \item \textbf{app.blade.php}\\
  Layout für sonstige Seiten ohne Sidebar
  \item \textbf{display.blade.php}\\
  Besonderes Layout für die Display Seiten
\end{itemize}

\begin{figure}[H]
  \centering
  \includegraphics[width=1\linewidth]{webinterface/admin_layout.pdf}
  \caption{Admin Layout}
\end{figure}

\begin{figure}[H]
  \centering
  \includegraphics[width=1\linewidth]{webinterface/app_layout.pdf}
  \caption{App Layout}
\end{figure}


\subsubsection{Navbar}


\subsubsection{Sidebar}


\subsubsection{Footer}

