Das Ziel dieser Arbeit ist es, eine Vereinfachung von Parkplatzüberwachungen zu
erstellen, welche möglichst überall eingesetzt werden kann. Dazu wird die Arbeit
in drei individuelle Teile aufgeteilt, welche einfach miteinander verbunden
werden können, um so einen modularen Aufbau bereitzustellen. Im ersten Teil geht
es um die Erkennung der Kennzeichen von Fahrzeugen. Dies wird beim Betreten und
Verlassen des Parkplatzes eingesetzt, um, festzustellen welches Fahrzeug sich im
Moment auf dem Parkplatz befindet. Zudem können dadurch unerlaubte Fahrzeuge
entdeckt werden und anschließend mögliche Schritte eingeleitet werden. Im
zweiten Teil geht es um die Erkennung von Fahrzeugen in einer Parklücke. Dazu
wird eine Spule unter jeder Parklücke verwendet, mit welcher festgestellt werden
kann, ob sich im Moment dort ein Fahrzeug befindet. Dadurch kann festgestellt
werden, welche Parkplätze besetzt sind und im Falle von mehrstöckigen Parkhäusern
kann die Auslastung der einzelnen Etagen überwacht werden. Im dritten Teil geht
es dann noch um die Darstellung und Verwaltung dieser Daten. Dazu wird eine
Web-Applikation bereitgestellt, welche alle Daten übersichtlich darstellt und
für den Parkplatzbetreiber und den Kunden unterschiedliche Benutzeroberflächen
bietet. Zudem kann die Web-Applikation mit wenig Aufwand von jedem
Parkplatzbetreiber personalisiert werden, was vor allem für Firmen interessant
ist, wenn sie diese in ihr Intranet einbauen möchten. Die Verbindung dieser drei
Teile erfolgt über eine eigens dafür geschriebenen API, wodurch die Daten über
das Internet übertragen werden können.