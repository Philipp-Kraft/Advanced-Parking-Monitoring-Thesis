\def \sectionauthors {Dennis Köb}
\subsection{Anforderungen}
Das Ziel der Fahrzeugerkennung ist es Fahrzeuge auf mehrere Parklücken eines Parkplatzes zu erkennen. Die daraus 
gewonnen Zustände sollen an das Webinterface übermittelt und an den jeweilige Parklücken über LEDs ausgegeben werden. 
\subsection{Vorstudie}

\subsection{Erkennung von Metallen über Spulen}
\subsubsection{Messung der Induktivität}
\paragraph{RL-Oszillator mit Timer Baustein}
\paragraph{Puls-Ladung mit Mikrokontroller}
\subsubsection{Messung der Induktion}
\paragraph{Messung eines Wechselstromes}
\paragraph{Messung der Resonanzfrequenz eines Oszillastors}

\subsection{RS485 Bussystem}
\subsubsection{Überblick}
\subsubsection{Elektrische Spezifikation}
\subsubsection{Implementation eines eigenen Protokolls}

\subsection{Mikrokontroller Slave-Geräte}
\subsubsection{Überblick}
\subsubsection{Atmega328PB}
\subsubsection{Peripherie des Mikrokontrollers}
\paragraph{Spannungswandler}
\paragraph{RS485 Pegelwandler}
\paragraph{Digitale Ein- und Ausgänge}
\subsubsection{Layout des Slave-Gerätes}
\subsubsection{Gehäuse}


\subsection{USB-Master}
\subsubsection{USB-Bussadapter Gerät}
\paragraph{Überblick}
\paragraph{FT232RL}
\paragraph{Spannungsversorgung}
\paragraph{USB-C Anschluss}
\paragraph{Layout des Master-Geräts}
\paragraph{Gehäuse}

\subsubsection{Master Programm}
\paragraph{Benötigte Software}
\paragraph{Adressvergabe}
\paragraph{Frequenzauslesung}
\paragraph{Auswertung}
\paragraph{API-Post}

\subsubsection{RaspberryPi als Mastergerät}
\paragraph{SSH Remote Zugriff}
\paragraph{Code Deployment}
\paragraph{Unittest}


